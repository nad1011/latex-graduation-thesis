\chapter{Đánh giá và thảo luận}

Chương 5, Đánh giá và thảo luận, tổng hợp và phân tích sâu hơn kết quả thực nghiệm từ việc triển khai các mẫu giao tiếp khác nhau trong kiến trúc vi dịch vụ. Chương này bắt đầu với việc mô tả tổng quan về phương pháp và tiêu chí đánh giá được áp dụng, bao gồm các chỉ số hiệu suất chính và công cụ đo lường được sử dụng trong quá trình thực nghiệm. Tiếp theo, chương trình bày chi tiết kết quả đánh giá cho từng kịch bản nghiệp vụ (Order-Inventory, Order-Payment, Order-Notification, và User Activity Logging), cùng với phân tích về ảnh hưởng của tải hệ thống đến hiệu suất của các mẫu giao tiếp. Phần thảo luận đi sâu vào việc phân tích các yếu tố quyết định trong lựa chọn mẫu giao tiếp phù hợp cho từng kịch bản cụ thể, đề xuất các mô hình tích hợp nhiều mẫu giao tiếp trong cùng một hệ thống, và các phương pháp tối ưu hóa hiệu suất trong môi trường thực tế. Chương cũng thảo luận về xu hướng phát triển và các hướng nghiên cứu tiếp theo trong lĩnh vực giao tiếp vi dịch vụ, đồng thời nhận diện các hạn chế của dự án hiện tại và đề xuất các hướng cải thiện trong tương lai.

\section{Phương pháp và tiêu chí đánh giá}

\subsection{Tổng quan phương pháp đánh giá}
Phương pháp đánh giá trong dự án này được thiết kế để cung cấp một cái nhìn toàn diện về hiệu suất và độ tin cậy của các mẫu giao tiếp khác nhau trong kiến trúc vi dịch vụ. Quá trình đánh giá tuân theo phương pháp luận khoa học nghiêm ngặt, bao gồm việc thiết lập các tiêu chí đánh giá rõ ràng, triển khai các kịch bản kiểm thử thực tế, và sử dụng các công cụ đo lường hiệu suất chuyên nghiệp.

Dự án xác định bốn kịch bản nghiệp vụ chính trong hệ thống thương mại điện tử: kiểm tra và cập nhật tồn kho, xử lý thanh toán, thông báo kết quả đơn hàng, và ghi nhận hoạt động người dùng. Mỗi kịch bản này đại diện cho một loại tương tác phổ biến trong các hệ thống vi dịch vụ, với các yêu cầu về hiệu suất và độ tin cậy khác nhau. Việc chọn các kịch bản đa dạng giúp đảm bảo kết quả đánh giá có tính đại diện cao và áp dụng được cho nhiều ngữ cảnh khác nhau.

Dự án triển khai ba mẫu giao tiếp chính: giao tiếp đồng bộ sử dụng REST API, giao tiếp bất đồng bộ dạng một-một sử dụng RabbitMQ, và giao tiếp bất đồng bộ dạng một-nhiều sử dụng Kafka. Mỗi mẫu giao tiếp được triển khai trên cùng một nền tảng hạ tầng và được đánh giá trong cùng các kịch bản, đảm bảo tính công bằng và khách quan của quá trình so sánh.

\subsection{Tiêu chí đánh giá cụ thể}
Các tiêu chí đánh giá được xây dựng dựa trên các yêu cầu phi chức năng quan trọng của hệ thống vi dịch vụ, phản ánh các khía cạnh chất lượng mà các kiến trúc vi dịch vụ hiện đại cần đáp ứng. Các tiêu chí này được phân thành bốn nhóm chính: hiệu suất, tài nguyên hệ thống, độ tin cậy và khả năng chịu lỗi.

Về hiệu suất, dự án đánh giá thời gian phản hồi (thời gian từ khi gửi yêu cầu đến khi nhận được phản hồi ban đầu), thời gian xử lý end-to-end (tổng thời gian để hoàn thành toàn bộ quy trình), thông lượng (số lượng yêu cầu xử lý trong một đơn vị thời gian), và thời gian phản hồi phân vị thứ 95 (P95).

Về tài nguyên hệ thống, dự án đánh giá mức sử dụng CPU, bộ nhớ và lưu lượng mạng. Các chỉ số này phản ánh hiệu quả sử dụng tài nguyên của mỗi mẫu giao tiếp, yếu tố quan trọng ảnh hưởng đến chi phí vận hành hệ thống.

Về độ tin cậy, dự án đánh giá tỷ lệ thành công, tỷ lệ lỗi và tỷ lệ nhất quán dữ liệu. Đặc biệt, tỷ lệ nhất quán dữ liệu đo lường khả năng duy trì tính nhất quán giữa các service, yếu tố quan trọng trong các hệ thống phân tán.

Về khả năng chịu lỗi, dự án đánh giá tỷ lệ lan truyền lỗi, thời gian phục hồi và tỷ lệ thành công một phần. Các chỉ số này phản ánh khả năng của hệ thống trong việc duy trì hoạt động khi gặp sự cố.

\subsection{Công cụ và môi trường đánh giá}
Dự án sử dụng một bộ công cụ hiện đại để thực hiện đánh giá. Công cụ k6 được sử dụng để mô phỏng lưu lượng người dùng và đo lường các chỉ số hiệu suất. Prometheus được sử dụng để thu thập và lưu trữ các số liệu về hiệu suất hệ thống. Các công cụ này đều là mã nguồn mở và được sử dụng rộng rãi trong cộng đồng phát triển phần mềm, đảm bảo tính tin cậy và khả năng mở rộng của quá trình đánh giá.

Môi trường kiểm thử được thiết kế để mô phỏng các điều kiện thực tế mà một hệ thống thương mại điện tử có thể gặp phải. Các kịch bản kiểm thử bao gồm các mức tải khác nhau, từ tải nhẹ đến tải nặng, để đánh giá hiệu suất của các mẫu giao tiếp trong nhiều điều kiện khác nhau. Các service được triển khai trên cùng một cấu hình phần cứng, và các bài kiểm thử được thực hiện nhiều lần để đảm bảo tính đại diện thống kê của kết quả.

Kịch bản kiểm thử được thiết kế để tự động thực hiện các tác vụ như tạo đơn hàng, kiểm tra tồn kho, xử lý thanh toán và gửi thông báo. Mỗi kịch bản được thực hiện trên những mẫu giao tiếp thích hợp với kịch bản đấy, và các chỉ số hiệu suất được ghi lại để so sánh.

Việc sử dụng các công cụ và môi trường đánh giá hiện đại giúp đảm bảo tính khách quan và chính xác của kết quả đánh giá. Điều này cho phép dự án đưa ra những kết luận đáng tin cậy về hiệu suất và độ tin cậy của các mẫu giao tiếp trong kiến trúc vi dịch vụ.

\section{Kết quả đánh giá}

\subsection{Mẫu giao tiếp trong kịch bản Order-Inventory}
Kết quả đánh giá hiệu suất giữa giao tiếp đồng bộ và bất đồng bộ trong kịch bản kiểm tra và cập nhật tồn kho cho thấy những đặc điểm đáng chú ý về cách các mẫu giao tiếp ảnh hưởng đến trải nghiệm người dùng và hiệu quả hệ thống. Giao tiếp bất đồng bộ thể hiện thời gian phản hồi ban đầu nhanh hơn đáng kể, giúp người dùng nhận được phản hồi tức thì khi thực hiện thao tác. Tuy nhiên, giao tiếp đồng bộ lại cho thời gian xử lý end-to-end nhỉnh hơn, phản ánh đặc tính xử lý trực tiếp, không qua trung gian của phương pháp này.

Về tính nhất quán dữ liệu, một phát hiện thú vị là giao tiếp bất đồng bộ đạt tỷ lệ nhất quán cao hơn, đặc biệt khi hệ thống chịu tải nặng. Điều này có vẻ trái ngược với quan niệm truyền thống rằng giao tiếp đồng bộ đảm bảo nhất quán dữ liệu tốt hơn. Nguyên nhân có thể do cơ chế hàng đợi giúp điều tiết luồng xử lý và tránh tình trạng quá tải, dẫn đến ít lỗi và nhất quán dữ liệu cao hơn.

Về hiệu quả sử dụng tài nguyên, giao tiếp bất đồng bộ thể hiện rõ ràng lợi thế với mức tiêu thụ CPU và bộ nhớ thấp hơn đáng kể. Điều này đặc biệt quan trọng trong môi trường đám mây, nơi chi phí vận hành thường tỷ lệ thuận với tài nguyên tiêu thụ.

\subsection{Mẫu giao tiếp trong kịch bản Order-Payment}
Trong kịch bản xử lý thanh toán, sự khác biệt về hiệu suất giữa giao tiếp đồng bộ và bất đồng bộ trở nên rõ rệt hơn. Giao tiếp bất đồng bộ thể hiện thời gian phản hồi ban đầu gần như tức thì, cải thiện đáng kể trải nghiệm người dùng so với phương pháp đồng bộ. Điều này đặc biệt quan trọng trong bối cảnh thanh toán, nơi người dùng mong đợi phản hồi nhanh chóng để biết yêu cầu của họ đã được tiếp nhận.

Thông lượng của giao tiếp bất đồng bộ vượt trội hơn hẳn so với giao tiếp đồng bộ, phản ánh khả năng tiếp nhận nhiều yêu cầu hơn trong cùng một khoảng thời gian. Điều này là do giao tiếp bất đồng bộ không bị chặn bởi thời gian xử lý của mỗi yêu cầu, cho phép hệ thống tiếp tục tiếp nhận yêu cầu mới trong khi các yêu cầu cũ đang được xử lý.

Đối với các trường hợp thanh toán kéo dài, giao tiếp bất đồng bộ vẫn duy trì thời gian phản hồi ban đầu thấp, trong khi giao tiếp đồng bộ buộc người dùng phải đợi đến khi toàn bộ quá trình hoàn tất. Điều này làm cho giao tiếp bất đồng bộ trở thành lựa chọn ưu việt hơn cho các tác vụ có thời gian xử lý dài như thanh toán.

\subsection{Mẫu giao tiếp trong kịch bản Order-Notification}
Kết quả đánh giá cho kịch bản thông báo kết quả đơn hàng thể hiện rõ ràng ưu điểm của mô hình Pub/Sub so với phương pháp gọi đồng bộ tuần tự. Mô hình Pub/Sub cung cấp thời gian broadcast và thời gian xử lý mỗi service nhanh hơn đáng kể, giúp thông báo được gửi đến tất cả các kênh nhanh chóng và hiệu quả.

Về khả năng chịu lỗi, mô hình Pub/Sub thể hiện thời gian phục hồi cực kỳ nhanh và tỷ lệ phục hồi thành công cao hơn so với phương pháp gọi đồng bộ tuần tự. Điều này phản ánh kiến trúc lỏng lẻo (loosely coupled) của mô hình Pub/Sub, nơi các subscriber hoạt động độc lập với nhau và với publisher, giúp hệ thống duy trì hoạt động ngay cả khi một số thành phần gặp sự cố.

Hiệu quả sử dụng tài nguyên của mô hình Pub/Sub cũng vượt trội hơn, với mức tiêu thụ CPU thấp hơn đáng kể. Điều này làm cho mô hình Pub/Sub trở thành lựa chọn tiết kiệm chi phí và hiệu quả hơn cho các kịch bản thông báo đa kênh.

\subsection{Mẫu giao tiếp trong kịch bản User Activity Logging}
Đối với kịch bản ghi nhận hoạt động người dùng, cả Kafka (mô hình một-nhiều) và RabbitMQ (mô hình một-một) đều thể hiện hiệu suất tương đương về thời gian phân phối và thông lượng. Tuy nhiên, Kafka sử dụng ít tài nguyên hơn và có thể xử lý nhiều consumer hơn một cách hiệu quả, làm cho nó phù hợp hơn cho các kịch bản có nhiều service cần truy cập cùng một dữ liệu.

Kafka cũng cung cấp khả năng lưu trữ dữ liệu dài hạn và phát lại các sự kiện, điều này đặc biệt hữu ích cho các kịch bản phân tích dữ liệu và học máy, nơi dữ liệu lịch sử có giá trị cao. RabbitMQ, mặt khác, phù hợp hơn cho các kịch bản yêu cầu điều hướng thông điệp phức tạp và định tuyến có điều kiện.

\subsection{Ảnh hưởng của tải hệ thống đến các mẫu giao tiếp}
Một khía cạnh quan trọng của dự án là đánh giá cách các mẫu giao tiếp hoạt động dưới các mức tải khác nhau. Kết quả cho thấy khi tải tăng, hiệu suất của cả giao tiếp đồng bộ và bất đồng bộ đều giảm, nhưng mức độ giảm khác nhau đáng kể.

Giao tiếp đồng bộ thể hiện sự suy giảm hiệu suất mạnh hơn khi tải tăng, với tỷ lệ nhất quán dữ liệu và thông lượng giảm nhanh. Điều này là do mô hình chặn (blocking model) của giao tiếp đồng bộ, nơi mỗi yêu cầu phải đợi đến khi yêu cầu trước đó hoàn thành. Khi số lượng yêu cầu tăng lên, điều này dẫn đến tình trạng nghẽn cổ chai và suy giảm hiệu suất.

Giao tiếp bất đồng bộ, mặt khác, duy trì hiệu suất ổn định hơn dưới tải nặng, nhờ vào khả năng đệm các yêu cầu và xử lý chúng theo tốc độ phù hợp với tài nguyên có sẵn. Đặc biệt, thời gian phản hồi ban đầu của giao tiếp bất đồng bộ vẫn duy trì ở mức thấp ngay cả khi tải tăng cao, giúp duy trì trải nghiệm người dùng tốt.

Mô hình Pub/Sub cũng thể hiện khả năng mở rộng tốt, với hiệu suất ổn định khi số lượng subscriber tăng lên. Điều này làm cho mô hình Pub/Sub trở thành lựa chọn phù hợp cho các hệ thống cần khả năng mở rộng theo chiều ngang.

\section{Thảo luận}

\subsection{Lựa chọn mẫu giao tiếp phù hợp cho từng kịch bản}
Dựa trên kết quả đánh giá toàn diện, có thể đưa ra những khuyến nghị về việc lựa chọn mẫu giao tiếp phù hợp trong kiến trúc vi dịch vụ.

Đối với kịch bản kiểm tra và cập nhật tồn kho, giao tiếp bất đồng bộ sử dụng Message Queue (RabbitMQ) được khuyến nghị vì thời gian phản hồi ban đầu nhanh (2.91ms so với 10.38ms), tỷ lệ nhất quán dữ liệu cao (97.2\% so với 93.9\%) và sử dụng tài nguyên hiệu quả (sử dụng CPU ít hơn 84\%). Mặc dù có thời gian xử lý end-to-end dài hơn một chút, nhưng ưu điểm này được bù đắp bởi khả năng duy trì hiệu suất ổn định dưới tải cao.

Đối với kịch bản xử lý thanh toán, giao tiếp bất đồng bộ sử dụng Message Queue cũng là lựa chọn ưu việt với thời gian phản hồi ban đầu nhanh hơn 99.8\% (2.25ms so với 1508.14ms) và thông lượng cao hơn 52 lần (90.04 msg/s so với 1.68 req/s). Trong bối cảnh thanh toán, việc phản hồi nhanh cho người dùng rằng yêu cầu đã được tiếp nhận là rất quan trọng, ngay cả khi quá trình xử lý thực sự có thể kéo dài.

Đối với kịch bản thông báo kết quả đơn hàng, mô hình Pub/Sub sử dụng Kafka là lựa chọn vượt trội với thời gian broadcast nhanh hơn 97.8\% (11.53ms so với 520.55ms), khả năng phục hồi tốt khi có service gặp lỗi (thời gian phục hồi 2.30ms so với 4793.46ms), và sử dụng tài nguyên hiệu quả (CPU thấp hơn 73.5\%). Mô hình này đặc biệt phù hợp khi một sự kiện cần được xử lý bởi nhiều service độc lập.

Đối với kịch bản ghi nhận hoạt động người dùng, Kafka (mô hình một-nhiều) được khuyến nghị vì hiệu quả sử dụng tài nguyên tốt hơn (CPU thấp hơn 15.8\% và bộ nhớ thấp hơn 10.5\%) và mô hình một-nhiều phù hợp hơn cho việc phân phối dữ liệu hoạt động đến nhiều service phân tích khác nhau.

\begin{table}[H]{Khuyến nghị mẫu giao tiếp phù hợp cho các kịch bản}
    \centering
    {\setlength{\arrayrulewidth}{1pt}
        \renewcommand{\arraystretch}{1.5}
        \setlength{\tabcolsep}{6pt}
        \begin{tabular}{|p{3.2cm}|p{3.2cm}|p{4.6cm}|}
            \hline
            \textbf{Mẫu giao tiếp}         & \textbf{Kịch bản áp dụng}                                       & \textbf{Lý do chính}                                                     \\
            \hline
            Đồng bộ (REST/GraphQL)         & Truy vấn đơn giản, thời gian xử lý ngắn, cần phản hồi trực tiếp & Tính nhất quán dữ liệu tức thì, dễ triển khai, thời gian end-to-end ngắn \\
            \hline
            Bất đồng bộ một-một (RabbitMQ) & Xử lý thời gian dài, cần phản hồi nhanh cho người dùng          & Phản hồi ban đầu nhanh, quản lý tải hiệu quả, khả năng chịu lỗi tốt      \\
            \hline
            Bất đồng bộ một-nhiều (Kafka)  & Phân phối sự kiện đến nhiều dịch vụ, cần lưu trữ dữ liệu        & Thời gian broadcast nhanh, dễ mở rộng, lưu trữ và phát lại sự kiện       \\
            \hline
        \end{tabular}}
\end{table}

Giao tiếp đồng bộ thích hợp cho những trường hợp cần tính nhất quán dữ liệu tức thì và phản hồi trực tiếp. Mẫu này có thời gian xử lý end-to-end ngắn hơn và thông lượng cao hơn trong một số tình huống, nhưng tiêu tốn nhiều tài nguyên và khó duy trì hiệu suất khi tải cao. Giao tiếp đồng bộ nên được lựa chọn khi cần đảm bảo tính nhất quán dữ liệu ngay lập tức, thời gian xử lý tác vụ ngắn và đơn giản, client cần phản hồi trực tiếp để tiếp tục quy trình nghiệp vụ, và khối lượng yêu cầu vừa phải và có thể dự đoán được.

Giao tiếp bất đồng bộ dạng một-một mang lại lợi thế về thời gian phản hồi ban đầu nhanh và sử dụng tài nguyên hiệu quả. Mẫu này giúp cải thiện trải nghiệm người dùng, đồng thời duy trì hiệu suất ổn định dưới tải cao. Mặc dù có thời gian xử lý end-to-end dài hơn một chút, nhưng ưu điểm này được bù đắp bởi khả năng chịu lỗi tốt hơn. Giao tiếp bất đồng bộ dạng một-một nên được lựa chọn khi cần phản hồi nhanh cho người dùng, thời gian xử lý tác vụ kéo dài, cần độ tin cậy cao trong xử lý message, cần quản lý tải và phân phối công việc hiệu quả, và hệ thống cần khả năng chịu lỗi và độ ổn định cao.

Giao tiếp bất đồng bộ dạng một-nhiều thể hiện hiệu suất vượt trội trong các tình huống cần phân phối sự kiện đến nhiều service. Mẫu này có thời gian broadcast nhanh, sử dụng tài nguyên hiệu quả, và khả năng phục hồi tuyệt vời khi có service gặp lỗi. Đặc biệt, mẫu này rất dễ mở rộng khi thêm các service xử lý mới. Giao tiếp bất đồng bộ dạng một-nhiều nên được lựa chọn khi một sự kiện cần được xử lý bởi nhiều service độc lập, cần khả năng lưu trữ và phát lại các sự kiện, hệ thống yêu cầu khả năng mở rộng cao, cần xử lý dữ liệu theo thời gian thực với thông lượng lớn, và các service xử lý có thể bị thêm hoặc xóa linh hoạt.

\subsection{Mô hình tích hợp các mẫu giao tiếp}
Trong thực tế, việc kết hợp các mẫu giao tiếp khác nhau thường mang lại kết quả tốt nhất cho hệ thống vi dịch vụ. Mỗi mẫu giao tiếp có những ưu điểm và nhược điểm riêng, phù hợp với những tình huống cụ thể. Một mô hình tích hợp hiệu quả có thể tận dụng ưu điểm của từng phương pháp để tối ưu hóa hiệu suất tổng thể của hệ thống.

Một chiến lược tích hợp hiệu quả là sử dụng giao tiếp đồng bộ cho các tác vụ yêu cầu phản hồi tức thời và có thời gian xử lý ngắn, như truy vấn thông tin đơn giản. Giao tiếp bất đồng bộ dạng một-một được sử dụng cho các tác vụ có thời gian xử lý dài nhưng cần phản hồi nhanh cho người dùng, như xử lý đơn hàng và thanh toán. Giao tiếp bất đồng bộ dạng một-nhiều được sử dụng cho các tác vụ cần phân phối thông tin đến nhiều service, như thông báo sự kiện và ghi nhận hoạt động.

Trong một hệ thống thương mại điện tử, mô hình tích hợp này có thể được triển khai như sau: REST API được sử dụng cho việc hiển thị thông tin sản phẩm và danh mục, RabbitMQ được sử dụng cho xử lý đơn hàng và thanh toán, và Kafka được sử dụng cho thông báo kết quả đơn hàng và phân tích dữ liệu.

Một yếu tố quan trọng cần xem xét khi thiết kế mô hình tích hợp là độ phức tạp của hệ thống. Việc sử dụng nhiều mẫu giao tiếp khác nhau có thể làm tăng độ phức tạp trong triển khai và bảo trì. Do đó, cần cân nhắc giữa lợi ích hiệu suất và chi phí phức tạp khi quyết định số lượng mẫu giao tiếp cần sử dụng.

\subsection{Tối ưu hóa hiệu suất trong thực tế}
Ngoài việc lựa chọn mẫu giao tiếp phù hợp, còn có nhiều chiến lược tối ưu hóa hiệu suất khác có thể áp dụng trong thực tế. Dựa trên kết quả đánh giá, có thể đưa ra một số khuyến nghị sau:

Đối với giao tiếp đồng bộ, việc triển khai Circuit Breaker pattern là rất quan trọng để ngăn chặn lỗi cascade và cải thiện khả năng chịu lỗi của hệ thống. Mẫu này giúp ngăn chặn các yêu cầu đến service không khả dụng, giảm thiểu tác động của lỗi dịch vụ đến toàn bộ hệ thống. Việc sử dụng timeouts hợp lý cũng giúp tránh tình trạng chờ đợi vô hạn khi service gặp sự cố.

Đối với giao tiếp bất đồng bộ, việc điều chỉnh kích thước hàng đợi và số lượng consumer có thể giúp cân bằng giữa thông lượng và độ trễ. Tăng số lượng consumer giúp cải thiện thông lượng, nhưng cũng làm tăng chi phí tài nguyên. Việc triển khai cơ chế retry với exponential backoff giúp xử lý các lỗi tạm thời, trong khi Dead Letter Queues giúp xử lý các thông điệp không thể xử lý.

Đối với mô hình Pub/Sub, việc phân vùng (partitioning) dữ liệu có thể giúp cải thiện khả năng mở rộng và hiệu suất. Việc chọn số lượng partition phù hợp với số lượng consumer giúp tối ưu hóa cân bằng tải và thông lượng. Việc duy trì kích thước thông điệp nhỏ và sử dụng định dạng nhị phân như Avro hoặc Protobuf thay vì JSON cũng giúp cải thiện hiệu suất.





