\chapter{Kiến thức nền tảng}

Chương 2, Kiến thức nền tảng, cung cấp nền tảng kiến thức toàn diện về kiến trúc vi dịch vụ và các cơ chế giao tiếp trong môi trường phân tán. Chương này bắt đầu với việc giới thiệu tổng quan về kiến trúc vi dịch vụ, bao gồm định nghĩa, đặc điểm nổi bật và so sánh với kiến trúc truyền thống. Tiếp theo, chương trình bày các lợi ích và thách thức khi áp dụng kiến trúc vi dịch vụ, cùng với các nguyên tắc thiết kế quan trọng để xây dựng hệ thống vi dịch vụ hiệu quả. Phần thứ hai của chương tập trung vào khái niệm và phân loại các mẫu giao tiếp trong kiến trúc vi dịch vụ, phân tích các đặc điểm, ưu nhược điểm của từng loại. Chương kết thúc với việc giới thiệu các công nghệ và giao thức giao tiếp phổ biến, cung cấp cái nhìn thực tế về cách triển khai các mẫu giao tiếp trong các hệ thống thực tế.

\section{Tổng quan về Microservice Architecture}

\subsection{Định nghĩa và đặc điểm}
Kiến trúc Microservice là một phương pháp phát triển phần mềm trong đó một ứng
dụng được cấu thành từ nhiều dịch vụ nhỏ, độc lập và có khả năng triển khai
riêng biệt. Mỗi dịch vụ này được thiết kế để thực hiện một chức năng cụ thể
trong phạm vi nghiệp vụ được định nghĩa rõ ràng, và giao tiếp với các dịch vụ
khác thông qua các cơ chế giao tiếp nhẹ, thường là API \cite{fowler2014}.

Các đặc điểm chính của kiến trúc microservice bao gồm tính tự trị cao, trong đó mỗi dịch vụ có thể được phát triển, triển khai và mở rộng độc lập với các dịch vụ khác \cite{newman2015}. Các dịch vụ được tổ chức xoay quanh các khả năng
nghiệp vụ thay vì các lớp công nghệ, thể hiện sự phân tách theo chức năng
nghiệp vụ. Quản lý dữ liệu trong microservice được thực hiện phi tập trung, với
mỗi dịch vụ quản lý dữ liệu riêng và chỉ có thể truy cập dữ liệu thông qua API
của dịch vụ sở hữu dữ liệu đó.

Thiết kế hướng lỗi là một đặc điểm quan trọng khác của microservice, trong đó
các dịch vụ được thiết kế để xử lý lỗi và khả năng các dịch vụ khác không khả
dụng. Cuối cùng, microservice cho phép tiến hóa độc lập, nghĩa là các dịch vụ
có thể thay đổi và phát triển theo thời gian mà không ảnh hưởng đến toàn bộ hệ
thống \cite{richardson2019}.

\subsection{So sánh với kiến trúc nguyên khối (Monolithic)}
Để hiểu rõ hơn về kiến trúc microservice, việc so sánh với kiến trúc nguyên khối là rất hữu ích. Trong kiến trúc nguyên khối, toàn bộ ứng dụng được xây dựng như một đơn vị duy nhất. Tất cả các chức năng nằm trong một codebase và được triển khai cùng nhau.

Về triển khai, kiến trúc nguyên khối đòi hỏi toàn bộ ứng dụng được triển khai
cùng một lúc, trong khi kiến trúc microservice cho phép các dịch vụ được triển
khai độc lập \cite{newman2015}. Điều này có ý nghĩa quan trọng trong việc giảm thiểu rủi ro và tăng tốc độ phát hành.

Khả năng mở rộng cũng khác biệt đáng kể giữa hai kiến trúc. Trong kiến trúc
nguyên khối, toàn bộ ứng dụng phải được mở rộng, ngay cả khi chỉ một phần cần
thêm tài nguyên. Ngược lại, kiến trúc microservice cho phép mở rộng từng dịch
vụ riêng biệt, tối ưu hóa việc sử dụng tài nguyên.

Về phát triển, kiến trúc nguyên khối thường có một nhóm phát triển làm việc
trên một codebase, dẫn đến các xung đột trong quá trình phát triển và triển
khai. Trong khi đó, kiến trúc microservice cho phép nhiều nhóm làm việc độc lập
trên các dịch vụ khác nhau, tăng tốc độ phát triển và giảm thiểu xung đột \cite{richardson2019}.

Công nghệ là một khía cạnh khác có sự khác biệt. Kiến trúc nguyên khối thường
bị giới hạn trong một stack công nghệ, trong khi mỗi microservice có thể sử
dụng công nghệ phù hợp nhất với yêu cầu của nó. Điều này tạo ra sự linh hoạt và
khả năng thích ứng với các công nghệ mới.

Khả năng chịu lỗi cũng là một điểm khác biệt quan trọng. Trong kiến trúc nguyên
khối, lỗi ở một phần có thể ảnh hưởng đến toàn bộ ứng dụng, trong khi trong
kiến trúc microservice, lỗi được cô lập trong một dịch vụ, giảm thiểu tác động
đến toàn bộ hệ thống \cite{fowler2014}.

Cuối cùng, về độ phức tạp, kiến trúc nguyên khối đơn giản hơn trong các ứng
dụng nhỏ, nhưng phức tạp hơn khi ứng dụng phát triển. Ngược lại, kiến trúc
microservice phức tạp hơn ngay từ đầu do tính phân tán, nhưng độ phức tạp này
được quản lý tốt hơn khi hệ thống phát triển.

\subsection{Lợi ích và thách thức của kiến trúc microservice}
Kiến trúc microservice mang lại nhiều lợi ích đáng kể cho việc phát triển và
vận hành phần mềm. Một trong những lợi ích chính là khả năng mở rộng có mục
tiêu. Các dịch vụ có thể được mở rộng độc lập dựa trên nhu cầu, tối ưu hóa việc
sử dụng tài nguyên. Điều này đặc biệt quan trọng trong môi trường cloud, nơi
chi phí tỷ lệ thuận với tài nguyên được sử dụng.

Phát triển nhanh hơn là một lợi ích khác của kiến trúc microservice. Các nhóm
nhỏ có thể làm việc trên các dịch vụ độc lập, cho phép phát triển song song và
chu kỳ phát hành nhanh hơn. Mỗi nhóm có thể tập trung vào một dịch vụ cụ thể,
hiểu rõ nó và phát triển nó một cách hiệu quả.

Tính linh hoạt công nghệ cũng là một lợi thế đáng kể. Mỗi dịch vụ có thể sử
dụng công nghệ phù hợp nhất với yêu cầu của nó. Ví dụ, một dịch vụ xử lý giao
dịch có thể sử dụng một ngôn ngữ chú trọng vào tính nhất quán, trong khi một
dịch vụ phân tích dữ liệu có thể sử dụng một ngôn ngữ tối ưu cho xử lý dữ liệu
lớn.

Khả năng chịu lỗi tốt hơn là một lợi ích khác của kiến trúc microservice. Lỗi
trong một dịch vụ không nhất thiết phải làm cho toàn bộ hệ thống không khả
dụng. Ví dụ, nếu dịch vụ gợi ý sản phẩm không hoạt động, người dùng vẫn có thể
duyệt và mua sản phẩm.

Khả năng bảo trì và hiểu biết tốt hơn cũng là một lợi thế của kiến trúc
microservice. Các dịch vụ nhỏ hơn dễ hiểu và bảo trì hơn các ứng dụng lớn. Mã
nguồn của mỗi dịch vụ nhỏ hơn và tập trung vào một chức năng cụ thể, giúp nhà
phát triển dễ dàng hiểu và thay đổi nó.

Tuy nhiên, kiến trúc microservice cũng đặt ra một số thách thức đáng kể. Độ
phức tạp phân tán là một thách thức lớn. Hệ thống phân tán vốn phức tạp hơn,
đòi hỏi kiến thức và công cụ chuyên biệt. Các vấn đề như latency mạng, xử lý
lỗi và đồng bộ hóa dữ liệu trở nên phức tạp hơn trong một hệ thống phân tán \cite{newman2015}.

Giao tiếp giữa các dịch vụ là một thách thức khác. Thiết kế và quản lý giao
tiếp giữa các dịch vụ đòi hỏi cân nhắc kỹ lưỡng về hiệu suất, độ tin cậy và khả
năng mở rộng. Việc lựa chọn giao thức giao tiếp phù hợp và xử lý các trường hợp
lỗi trong giao tiếp là các vấn đề phức tạp.

Quản lý dữ liệu cũng là một thách thức đáng kể trong kiến trúc microservice.
Duy trì tính nhất quán dữ liệu giữa các dịch vụ có thể phức tạp, đặc biệt là
khi mỗi dịch vụ có cơ sở dữ liệu riêng. Các mẫu như Saga và Event Sourcing được
sử dụng để giải quyết vấn đề này, nhưng chúng cũng đưa ra sự phức tạp riêng.

Vận hành và giám sát là một thách thức khác của kiến trúc microservice. Triển
khai và giám sát nhiều dịch vụ đòi hỏi công cụ và quy trình tinh vi hơn. Các
công cụ như Kubernetes và Prometheus đã được phát triển để giải quyết vấn đề
này, nhưng chúng cũng đòi hỏi kiến thức và nỗ lực đáng kể để sử dụng hiệu quả.

Cuối cùng, kiểm thử cũng trở nên phức tạp hơn trong kiến trúc microservice.
Kiểm thử tích hợp đòi hỏi sự phối hợp giữa nhiều dịch vụ, có thể chạy trên các
máy khác nhau và sử dụng các công nghệ khác nhau. Các kỹ thuật như kiểm thử hợp
đồng và môi trường kiểm thử tích hợp được sử dụng để giải quyết vấn đề này \cite{newman2015}.

\subsection{Các nguyên tắc thiết kế}
Để thiết kế một kiến trúc microservice hiệu quả, một số nguyên tắc thiết kế chính cần được tuân thủ. Nguyên tắc đầu tiên là Single Responsibility Principle (Nguyên tắc Trách nhiệm Đơn lẻ), theo đó mỗi dịch vụ nên chịu trách nhiệm cho một chức năng nghiệp vụ duy nhất. Điều này giúp giữ các dịch vụ đơn giản và tập trung, dễ hiểu và bảo trì.

Domain-Driven Design (DDD) là một phương pháp thiết kế hữu ích cho kiến trúc
microservice. DDD sử dụng các khái niệm như Bounded Context để định nghĩa ranh
giới giữa các dịch vụ. Bounded Context giúp xác định phạm vi trách nhiệm của
mỗi dịch vụ và cách chúng tương tác với nhau.

API First là một nguyên tắc khác, nhấn mạnh việc thiết kế API trước, xem nó như
một hợp đồng giữa các dịch vụ. Điều này giúp đảm bảo rằng các dịch vụ có thể
giao tiếp hiệu quả và rằng các thay đổi không phá vỡ tương thích ngược.

Tự động hóa là một phần quan trọng của kiến trúc microservice thành công. Tự
động hóa quá trình xây dựng, kiểm thử và triển khai giúp quản lý sự phức tạp
của việc phát triển và vận hành nhiều dịch vụ. Các công cụ CI/CD (Continuous
Integration/Continuous Deployment) là rất quan trọng trong môi trường
microservice.

Monitoring và Observability là các nguyên tắc quan trọng khác. Thiết kế hệ
thống để dễ dàng giám sát và hiểu được hoạt động nội bộ giúp phát hiện và giải
quyết vấn đề một cách nhanh chóng. Các công cụ như logging tập trung, theo dõi
phân tán và thu thập số liệu là rất quan trọng.

Cuối cùng, Fault Tolerance (Khả năng chịu lỗi) là một nguyên tắc thiết kế quan
trọng cho kiến trúc microservice. Các dịch vụ nên được thiết kế để xử lý lỗi
một cách thanh nhã, sử dụng các kỹ thuật như Circuit Breaker. Circuit Breaker
ngăn lỗi lan truyền bằng cách ngừng gửi yêu cầu đến các dịch vụ không phản hồi.
\section{Giao tiếp trong kiến trúc vi dịch vụ}

\subsection{Vai trò của giao tiếp trong kiến trúc vi dịch vụ}
Giao tiếp trong kiến trúc vi dịch vụ vượt xa khái niệm đơn giản về việc truyền dữ liệu từ điểm A đến điểm B. Nó là xương sống kết nối các thành phần độc lập, định hình cách thức vận hành của toàn hệ thống và trực tiếp ảnh hưởng đến những thuộc tính quan trọng như tính khả dụng, hiệu suất và khả năng mở rộng.

Hãy tưởng tượng một quy trình đặt hàng trực tuyến. Để hoàn thành một đơn hàng, nhiều dịch vụ phải phối hợp: dịch vụ quản lý đơn hàng, dịch vụ thanh toán, dịch vụ kho hàng và dịch vụ vận chuyển. Chỉ khi các dịch vụ này giao tiếp hiệu quả, quy trình mới diễn ra suôn sẻ. Một lỗi giao tiếp duy nhất có thể dẫn đến vấn đề nghiêm trọng như đơn hàng không được xử lý, thanh toán không thành công, hoặc hàng hóa không được gửi đi.

Trong môi trường dữ liệu phân tán của vi dịch vụ, mỗi dịch vụ quản lý một phần dữ liệu riêng biệt. Khi dữ liệu thay đổi, giao tiếp là phương tiện duy nhất để đảm bảo tính nhất quán trên toàn hệ thống. Ví dụ, khi thông tin khách hàng được cập nhật trong dịch vụ quản lý người dùng, các dịch vụ khác cần được thông báo để phản ánh thay đổi này.

Giao tiếp còn đóng vai trò quan trọng trong việc đảm bảo khả năng chịu lỗi của hệ thống. Các cơ chế như Circuit Breaker giúp ngăn chặn lỗi lan truyền, cho phép hệ thống tiếp tục hoạt động ngay cả khi một số thành phần gặp sự cố. Thiết kế giao tiếp tốt cũng tạo điều kiện để mở rộng hệ thống một cách linh hoạt, cho phép thêm dịch vụ mới hoặc nâng cấp phiên bản mà không ảnh hưởng đến các dịch vụ hiện có.

\subsection{Các thuộc tính quan trọng của giao tiếp vi dịch vụ}
Khi thiết kế giao tiếp cho vi dịch vụ, chúng ta cần cân nhắc nhiều thuộc tính then chốt tạo nên một hệ thống mạnh mẽ và linh hoạt.

Độ tin cậy là nền tảng của mọi hệ thống giao tiếp. Trong môi trường phân tán, thông điệp có thể bị mất, bị trễ hoặc bị trùng lặp. Các cơ chế như xác nhận, thử lại tự động và hàng đợi bền vững giúp đảm bảo mọi thông điệp đều được xử lý đúng cách, ngay cả khi có sự cố xảy ra. Không chỉ đơn thuần là "gửi và quên", giao tiếp đáng tin cậy đòi hỏi những giải pháp toàn diện để xử lý các tình huống không mong muốn.

Độ trễ, hay thời gian từ khi thông điệp được gửi đến khi nhận, ảnh hưởng trực tiếp đến trải nghiệm người dùng và hiệu suất hệ thống. Khoảng cách vật lý giữa các dịch vụ, phương thức tuần tự hóa dữ liệu, và tải mạng đều là những yếu tố tác động đến độ trễ. Việc giảm thiểu độ trễ thường đòi hỏi sự đánh đổi với các thuộc tính khác, đặc biệt là độ tin cậy và tính nhất quán.

Khả năng mở rộng cho phép hệ thống xử lý khối lượng thông điệp ngày càng tăng khi doanh nghiệp phát triển. Một hệ thống mở rộng tốt không chỉ đơn giản là thêm nhiều máy chủ, mà còn phải thiết kế để phân phối tải một cách hiệu quả giữa các phiên bản dịch vụ, tránh nghẽn cổ chai và duy trì hiệu suất ổn định.

Cách ly lỗi là khả năng ngăn chặn lỗi từ một dịch vụ lan truyền sang các dịch vụ khác, gây ra hiệu ứng domino. Các mẫu thiết kế như Circuit Breaker và Bulkhead giúp hạn chế phạm vi ảnh hưởng của lỗi, cho phép hệ thống tiếp tục hoạt động ngay cả khi một số thành phần gặp sự cố. Không chỉ là khả năng phục hồi sau lỗi, cách ly lỗi còn là khả năng duy trì chức năng cốt lõi trong điều kiện không lý tưởng.

Tính nhất quán liên quan đến cách thức đảm bảo dữ liệu nhất quán trong một hệ thống phân tán. Theo định lý CAP, không thể đồng thời đảm bảo tính nhất quán, khả năng sẵn sàng và khả năng chịu đựng phân vùng. Vi dịch vụ thường hy sinh tính nhất quán tức thời để đạt được khả năng sẵn sàng cao, áp dụng mô hình nhất quán cuối cùng (eventual consistency) thay vì giao dịch phân tán truyền thống.

Định dạng dữ liệu quyết định cách thông tin được cấu trúc và tuần tự hóa khi truyền giữa các dịch vụ. JSON và XML là các định dạng văn bản phổ biến, dễ đọc và linh hoạt, trong khi Protocol Buffers và Avro cung cấp hiệu suất cao hơn nhờ định dạng nhị phân. Việc lựa chọn định dạng cần cân nhắc giữa hiệu suất, khả năng tương tác và dễ sử dụng.

Khả năng tương tác cho phép các dịch vụ sử dụng công nghệ khác nhau giao tiếp hiệu quả. Trong môi trường đa ngôn ngữ và đa nền tảng, khả năng tương tác trở nên đặc biệt quan trọng, đòi hỏi các giao thức chuẩn và định dạng dữ liệu được hỗ trợ rộng rãi. Điều này tạo điều kiện cho các đội phát triển độc lập lựa chọn công nghệ phù hợp nhất cho từng dịch vụ.

Bảo mật luôn là mối quan tâm hàng đầu trong bất kỳ hệ thống nào. Khi dịch vụ giao tiếp qua mạng, thông điệp có thể bị đánh chặn, giả mạo hoặc thay đổi. Mã hóa bảo vệ tính bảo mật của dữ liệu, trong khi xác thực và ủy quyền đảm bảo chỉ các bên được phép mới có thể tham gia giao tiếp.

\subsection{Các mô hình giao tiếp cơ bản}
Hai mô hình giao tiếp cơ bản trong kiến trúc vi dịch vụ - đồng bộ và bất đồng bộ - định hình cách thức tương tác giữa các dịch vụ và ảnh hưởng sâu sắc đến thiết kế hệ thống tổng thể.

Trong giao tiếp đồng bộ, dịch vụ gửi yêu cầu và chờ đợi phản hồi trước khi tiếp tục xử lý. Giống như cuộc đối thoại trực tiếp, người gửi tạm dừng hoạt động của mình để đợi phản hồi. Ví dụ, khi dịch vụ đơn hàng gửi yêu cầu kiểm tra tồn kho đến dịch vụ kho hàng, nó sẽ đợi xác nhận trước khi chấp nhận đơn hàng. Mô hình này đơn giản, dễ hiểu và cung cấp phản hồi tức thì, giúp duy trì tính nhất quán dữ liệu. Tuy nhiên, nó có thể dẫn đến hiệu suất kém vì dịch vụ phải chờ đợi trong trạng thái không hoạt động. Hơn nữa, nếu dịch vụ nhận yêu cầu chậm hoặc không phản hồi, có thể gây ra hiệu ứng dây chuyền ảnh hưởng đến toàn bộ hệ thống.

Ngược lại, giao tiếp bất đồng bộ hoạt động như gửi email - người gửi không cần chờ đợi phản hồi ngay lập tức. Dịch vụ gửi thông điệp vào hàng đợi hoặc kênh rồi tiếp tục xử lý công việc khác. Dịch vụ nhận xử lý thông điệp khi có khả năng và có thể gửi phản hồi thông qua một kênh riêng biệt. Mô hình này tạo ra sự liên kết lỏng lẻo giữa các dịch vụ, cải thiện khả năng chịu lỗi và mở rộng. Nếu một dịch vụ tạm thời không khả dụng, các thông điệp vẫn được lưu trong hàng đợi để xử lý sau. Tuy nhiên, giao tiếp bất đồng bộ phức tạp hơn để triển khai, có thể dẫn đến độ trễ cao, và gây khó khăn trong việc theo dõi luồng xử lý cũng như đảm bảo tính nhất quán dữ liệu.

\subsection{Kiểu tương tác}
Bên cạnh mô hình giao tiếp, kiểu tương tác giữa các dịch vụ cũng đóng vai trò quan trọng trong việc định hình kiến trúc tổng thể.

Kiểu one-to-one (một-một) là hình thức giao tiếp trực tiếp giữa hai dịch vụ cụ thể. Giống như cuộc đối thoại riêng tư, thông điệp được gửi từ một nguồn đến một đích xác định. Ví dụ, khi dịch vụ đơn hàng cần xử lý thanh toán, nó gửi yêu cầu trực tiếp đến dịch vụ thanh toán và nhận phản hồi cụ thể. Kiểu tương tác này thường được triển khai thông qua REST API hoặc RPC, tạo ra mối quan hệ rõ ràng và trực tiếp giữa các dịch vụ. Mặc dù đơn giản và dễ quản lý, phương pháp này có thể dẫn đến sự phụ thuộc chặt chẽ và khó mở rộng khi số lượng dịch vụ tăng lên.

Kiểu one-to-many (một-nhiều) là khi một dịch vụ cần truyền thông tin đến nhiều dịch vụ khác cùng một lúc. Tương tự như thông báo công khai, thông điệp được phát ra và bất kỳ dịch vụ quan tâm nào cũng có thể nhận. Ví dụ, khi dịch vụ đơn hàng xác nhận đơn hàng mới, nó có thể phát một sự kiện để thông báo cho nhiều dịch vụ như kho hàng, vận chuyển, thông báo khách hàng và phân tích dữ liệu. Kiểu này thường được triển khai thông qua mô hình publish/subscribe sử dụng message broker hoặc event bus. Phương pháp này tạo ra sự kết nối lỏng lẻo, cho phép thêm hoặc thay đổi người nhận mà không ảnh hưởng đến người gửi, tăng tính linh hoạt và khả năng mở rộng. Tuy nhiên, nó cũng làm tăng độ phức tạp trong việc theo dõi luồng dữ liệu và đảm bảo tính nhất quán.

Sự kết hợp giữa mô hình giao tiếp (đồng bộ/bất đồng bộ) và kiểu tương tác (một-một/một-nhiều) tạo ra các mẫu giao tiếp đa dạng, mỗi mẫu đều có các ưu nhược điểm riêng và phù hợp với các tình huống sử dụng cụ thể trong kiến trúc vi dịch vụ.

\subsection{Các công nghệ và giao thức phổ biến}
Kiến trúc vi dịch vụ hiện đại cung cấp nhiều lựa chọn công nghệ, mỗi công nghệ đều có đặc điểm và ưu điểm riêng biệt phù hợp với các yêu cầu cụ thể.

HTTP/REST là lựa chọn phổ biến nhất cho giao tiếp đồng bộ giữa các vi dịch vụ. Sử dụng các phương thức HTTP tiêu chuẩn (GET, POST, PUT, DELETE) và tài nguyên được định danh bằng URL, REST cung cấp một mô hình đơn giản và trực quan. Ví dụ, để truy vấn thông tin sản phẩm, dịch vụ có thể gửi yêu cầu GET đến đường dẫn /products/{id}. Đơn giản, được hỗ trợ rộng rãi và không trạng thái, REST là lựa chọn tự nhiên cho nhiều dự án. Tuy nhiên, nó không phải lúc nào cũng là giải pháp hiệu quả nhất về mặt hiệu suất, đặc biệt khi cần truyền dữ liệu lớn hoặc tương tác phức tạp.

gRPC là framework RPC hiệu suất cao phát triển bởi Google, sử dụng HTTP/2 làm giao thức vận chuyển và Protocol Buffers cho tuần tự hóa dữ liệu. Không như REST với các yêu cầu và phản hồi đơn giản, gRPC cho phép xác định các dịch vụ với nhiều phương thức có thể được gọi từ xa. Bằng cách tận dụng multiplexing trên một kết nối TCP duy nhất và định dạng nhị phân hiệu quả, gRPC cung cấp hiệu suất cao hơn đáng kể so với REST, đặc biệt trong môi trường độ trễ cao. Nó cũng hỗ trợ streaming hai chiều, lý tưởng cho các trường hợp như theo dõi dữ liệu thời gian thực. Mặc dù mạnh mẽ, gRPC có độ phức tạp cao hơn và không được hỗ trợ trực tiếp bởi tất cả các nền tảng, đặc biệt là trình duyệt web.
Message Queue là nền tảng của giao tiếp bất đồng bộ, cho phép các dịch vụ gửi và nhận thông điệp thông qua hàng đợi. Các hệ thống phổ biến như RabbitMQ, ActiveMQ và AWS SQS cung cấp các cơ chế khác nhau cho việc định tuyến và xử lý thông điệp. Ví dụ, khi đơn hàng được tạo, dịch vụ đơn hàng có thể đặt thông điệp vào hàng đợi để dịch vụ kho hàng xử lý khi có sẵn tài nguyên. Message Queue cung cấp sự kết nối lỏng lẻo, khả năng đệm và độ tin cậy cao, nhưng cũng thêm độ phức tạp và độ trễ vào hệ thống.

Publish/Subscribe (Pub/Sub) là mô hình giao tiếp bất đồng bộ nâng cao, trong đó người gửi (nhà xuất bản) không biết về người nhận cụ thể (người đăng ký). Thay vào đó, thông điệp được phát ra cho một chủ đề, và bất kỳ dịch vụ nào quan tâm đều có thể đăng ký để nhận. Giải pháp như Apache Kafka, AWS SNS/SQS, Google Pub/Sub và NATS mỗi hệ thống đều có đặc điểm riêng về khả năng mở rộng, độ trễ và độ bền. Pub/Sub đặc biệt phù hợp với các trường hợp một sự kiện cần được xử lý bởi nhiều dịch vụ độc lập, như khi một đơn hàng được xác nhận cần thông báo cho kho hàng, vận chuyển và thông báo khách hàng.

GraphQL là phương pháp tiếp cận mới hơn, hoạt động như một lớp truy vấn thống nhất trên nhiều dịch vụ. Không giống như REST với các endpoint cố định, GraphQL cho phép client chỉ định chính xác dữ liệu cần thiết, tránh over-fetching và under-fetching. Chẳng hạn, một ứng dụng di động có thể yêu cầu chỉ những trường cụ thể của thông tin sản phẩm thay vì toàn bộ đối tượng. Điều này đặc biệt hữu ích cho các ứng dụng di động với băng thông hạn chế, nhưng đòi hỏi thiết kế schema cẩn thận và có thể gặp khó khăn với các truy vấn phức tạp.

\subsection{Thách thức trong giao tiếp vi dịch vụ}
Mặc dù mang lại nhiều lợi ích, giao tiếp vi dịch vụ cũng đặt ra những thách thức đáng kể cần được giải quyết để xây dựng hệ thống đáng tin cậy.

Network Reliability là thách thức nền tảng khi các dịch vụ phải giao tiếp qua mạng vốn không hoàn hảo. Mạng có thể chậm, không ổn định hoặc tạm thời không khả dụng, dẫn đến mất thông điệp, độ trễ cao hoặc timeout. Để đối phó, các hệ thống phải triển khai các cơ chế như retry với exponential backoff, timeout hợp lý và circuit breaker để ngăn lỗi lan truyền. Những giải pháp này cần được thiết kế cẩn thận để tránh tạo thêm vấn đề, như hiệu ứng "thundering herd" khi nhiều dịch vụ đồng loạt thử kết nối lại sau lỗi.

Service Discovery giải quyết câu hỏi làm thế nào các dịch vụ có thể tìm thấy nhau trong môi trường liên tục thay đổi. Với các dịch vụ được triển khai, di chuyển hoặc mở rộng thường xuyên, địa chỉ IP và cổng không còn cố định. Các giải pháp như Client-side Discovery sử dụng service registry (như Netflix Eureka, Consul) để lưu trữ thông tin về các instance khả dụng, trong khi Server-side Discovery dùng load balancer hoặc API gateway để định tuyến yêu cầu. Mỗi phương pháp đều có đánh đổi giữa độ phức tạp, hiệu suất và khả năng chịu lỗi.

Load Balancing đảm bảo rằng yêu cầu được phân phối hiệu quả giữa các instance của dịch vụ, tránh quá tải bất kỳ instance nào. Các thuật toán như Round Robin, Least Connections, và Hash-based Routing phân phối tải dựa trên các tiêu chí khác nhau. Hệ thống hiện đại thường kết hợp nhiều kỹ thuật, như Service Mesh (Istio, Linkerd) điều khiển traffic ở lớp network, cung cấp các tính năng nâng cao như phát hiện lỗi, retry tự động và canary deployment.

Data Consistency là một trong những thách thức phức tạp nhất trong kiến trúc phân tán. Khi không có cơ sở dữ liệu chung, làm thế nào để đảm bảo các thao tác trải rộng qua nhiều dịch vụ hoàn thành nhất quán? Các mẫu như Saga chia giao dịch thành các bước nhỏ hơn với cơ chế bù trừ, Event Sourcing lưu trữ chuỗi các sự kiện thay vì trạng thái hiện tại, và CQRS tách biệt mô hình đọc và ghi. Những giải pháp này cho phép tính nhất quán cuối cùng thay vì nhất quán tức thời, đánh đổi giữa tính nhất quán và khả năng sẵn sàng.

Versioning là thách thức đối với việc phát triển và triển khai các dịch vụ một cách độc lập. Khi API thay đổi, làm thế nào để tránh phá vỡ các dịch vụ khác phụ thuộc vào nó? Semantic Versioning với hệ thống đánh số rõ ràng (MAJOR.MINOR.PATCH) giúp truyền đạt mức độ thay đổi. API Versioning duy trì nhiều phiên bản API cùng lúc, cho phép client di chuyển dần dần sang phiên bản mới. Backward Compatibility đảm bảo các thay đổi không làm hỏng client hiện có, thường bằng cách thêm các trường tùy chọn thay vì loại bỏ hoặc thay đổi các trường hiện có.

Error Handling trong hệ thống phân tán phức tạp hơn nhiều so với ứng dụng monolithic. Lỗi có thể xảy ra ở bất kỳ điểm nào trong chuỗi các dịch vụ, và hệ thống cần có cơ chế để phát hiện, xử lý và phục hồi từ các lỗi này. Retry thử lại các yêu cầu thất bại với backoff và jitter để tránh quá tải. Circuit Breaker ngăn yêu cầu đến dịch vụ không phản hồi, cho phép nó phục hồi và tránh hiệu ứng cascade. Fallback cung cấp phản hồi thay thế khi dịch vụ không khả dụng, như dữ liệu được cache hoặc giá trị mặc định.

Monitoring and Debugging trong hệ thống vi dịch vụ đòi hỏi phương pháp tiếp cận toàn diện. Một yêu cầu đơn lẻ có thể đi qua nhiều dịch vụ, làm cho việc theo dõi và gỡ lỗi trở nên phức tạp. Logging tập trung thu thập log từ tất cả dịch vụ vào một nơi để phân tích. Distributed Tracing (như Jaeger, Zipkin) theo dõi yêu cầu qua nhiều dịch vụ, cung cấp cái nhìn toàn cảnh về hiệu suất và lỗi. Metrics Collection thu thập dữ liệu về hiệu suất hệ thống, từ mức sử dụng tài nguyên đến thời gian phản hồi và tỷ lệ lỗi, cho phép phát hiện sớm vấn đề.

\subsection{Các mẫu giao tiếp}
Trong thực tế, kiến trúc vi dịch vụ thường sử dụng kết hợp các mẫu giao tiếp khác nhau để giải quyết các yêu cầu đa dạng của hệ thống.

Request-Response là mẫu đồng bộ phổ biến nhất, dịch vụ gửi yêu cầu đến dịch vụ khác và đợi phản hồi. Dịch vụ gửi thiết lập kết nối HTTP/REST hoặc gRPC, chờ đợi phản hồi từ dịch vụ nhận. Mẫu này đơn giản, dễ hiểu và đảm bảo tính nhất quán dữ liệu cao. Tuy nhiên, tạo coupling chặt chẽ, hiệu suất kém khi độ trễ cao, và có nguy cơ lỗi cascade.

Event-Driven là mẫu các dịch vụ giao tiếp qua phát và lắng nghe sự kiện thông qua message broker. Dịch vụ phát hành không cần biết ai xử lý sự kiện, tạo decoupling cao và khả năng mở rộng tốt. Tuy nhiên, việc theo dõi luồng thực thi và gỡ lỗi phức tạp hơn, khó duy trì tính nhất quán dữ liệu.

Publish-Subscribe là dạng cụ thể của Event-Driven, cho phép phân phối thông tin từ một nguồn đến nhiều người nhận. Publisher gửi thông điệp đến kênh, nhiều subscribers nhận từ kênh đó. Triển khai qua Apache Kafka, RabbitMQ hoặc NATS. Phù hợp cho truyền thông tin một-đến-nhiều, dễ mở rộng, nhưng phức tạp trong quản lý tính nhất quán và có thể xử lý trùng lặp.

Point-to-Point Messaging gửi thông điệp từ nguồn đến đích cụ thể qua hàng đợi. Producer gửi thông điệp vào hàng đợi, chỉ một consumer xử lý mỗi thông điệp. Đảm bảo tin cậy cao, phù hợp cho phân phối tác vụ và cân bằng tải. Tuy nhiên, có thể nghẽn hàng đợi nếu xử lý chậm và không phù hợp khi nhiều dịch vụ cần nhận cùng thông tin.

Asynchronous Request-Response là biến thể bất đồng bộ của Request-Response. Dịch vụ gửi yêu cầu và tiếp tục xử lý, dịch vụ nhận xử lý và gửi phản hồi qua hàng đợi. Dịch vụ gửi được thông báo qua callback, webhook hoặc long polling. Tránh blocking, cải thiện hiệu suất, nhưng phức tạp hơn trong triển khai và quản lý.

Mỗi mẫu có ưu nhược điểm riêng, lựa chọn phù hợp phụ thuộc vào yêu cầu về tính nhất quán, hiệu suất, khả năng mở rộng và độ tin cậy. Thực tế, hệ thống vi dịch vụ thường kết hợp nhiều mẫu để giải quyết các tình huống khác nhau hiệu quả.
\section{Công nghệ và phương pháp đo lường hiệu năng}

\subsection{Các công nghệ triển khai trong dự án}
Trong triển khai kiến trúc vi dịch vụ, việc lựa chọn công nghệ phù hợp đóng vai trò quan trọng, ảnh hưởng trực tiếp đến hiệu suất, khả năng mở rộng và bảo trì của hệ thống \cite{newman2015}. Khóa luận sử dụng NestJS làm framework chính cho việc phát triển microservices, một framework Node.js tiến bộ dựa trên TypeScript, cung cấp kiến trúc ứng dụng lấy cảm hứng từ Angular với các nguyên tắc SOLID và mô hình MVC. Framework này mang lại lợi ích như hỗ trợ dependency injection, kiến trúc mô-đun hóa cao và tích hợp sẵn với nhiều công nghệ khác.

Mỗi dịch vụ được triển khai như một ứng dụng NestJS độc lập, với cấu trúc gồm controllers (xử lý yêu cầu HTTP), services (chứa logic nghiệp vụ), modules (đóng gói thành phần liên quan) và entities (đại diện đối tượng dữ liệu). NestJS cung cấp module microservices chuyên dụng hỗ trợ các giao thức như TCP, Redis, MQTT, gRPC, và Kafka, giúp đơn giản hóa việc triển khai các mẫu giao tiếp.

TypeScript được chọn làm ngôn ngữ lập trình chính với ưu điểm hệ thống kiểu dữ liệu tĩnh, giúp phát hiện lỗi sớm, tăng cường khả năng đọc hiểu và bảo trì mã nguồn. Trong môi trường microservices, TypeScript giúp đảm bảo tính nhất quán của dữ liệu được truyền giữa các dịch vụ thông qua các contract rõ ràng.

Về lưu trữ dữ liệu, nguyên tắc "mỗi dịch vụ có cơ sở dữ liệu riêng" được tuân thủ. TypeORM được sử dụng để tương tác với cơ sở dữ liệu, hỗ trợ nhiều hệ quản trị và cung cấp tính năng như quan hệ, kế thừa, migrations. TypeORM sử dụng cả Active Record và Data Mapper, hỗ trợ lazy/eager loading, transactions và query builder để tối ưu hóa hiệu suất truy vấn.

PostgreSQL được chọn làm hệ quản trị cơ sở dữ liệu chính do tính ổn định, hiệu suất cao và hỗ trợ dữ liệu phức tạp (JSON, JSONB, arrays). Khả năng xử lý đồng thời và transaction của PostgreSQL đảm bảo tính nhất quán dữ liệu trong môi trường phân tán.

Về giao tiếp giữa microservices, khóa luận sử dụng nhiều công nghệ cho các mẫu giao tiếp khác nhau. HTTP/REST API là nền tảng cho giao tiếp đồng bộ, với Axios làm HTTP client. RabbitMQ được triển khai cho mẫu Point-to-Point và Asynchronous Request-Response, cung cấp cơ chế tin cậy cao với xác nhận tin nhắn và hàng đợi bền vững. Apache Kafka được sử dụng cho Publish/Subscribe và Event-Driven, nổi bật với khả năng xử lý hàng triệu sự kiện mỗi giây, độ trễ thấp và lưu trữ sự kiện lâu dài.

\subsection{Các thông số đo lường chính}
Để đánh giá hiệu năng của các mẫu giao tiếp, khóa luận xem xét một tập hợp thông số toàn diện. Latency (Độ trễ) là thông số quan trọng nhất, đại diện cho thời gian cần thiết để hoàn thành một yêu cầu, từ khi gửi đến khi nhận phản hồi \cite{jun2018}. Độ trễ được phân tích theo nhiều khía cạnh: độ trễ đầu cuối (tổng thời gian từ client đến phản hồi), độ trễ dịch vụ (thời gian xử lý trong một microservice) và độ trễ mạng (thời gian di chuyển dữ liệu giữa dịch vụ).

Throughput (Thông lượng) đo lường số lượng yêu cầu hệ thống xử lý trong một đơn vị thời gian, biểu thị bằng yêu cầu/giây (RPS) hoặc giao dịch/giây (TPS) \cite{jun2018}. Thông lượng được đo ở nhiều cấp độ: hệ thống, dịch vụ và endpoint. Các mẫu giao tiếp khác nhau ảnh hưởng đáng kể đến thông lượng - mẫu đồng bộ thường có thông lượng thấp hơn, mẫu bất đồng bộ có thể đạt thông lượng cao hơn nhờ xử lý song song.

Error Rate (Tỷ lệ lỗi) là tỷ lệ phần trăm yêu cầu thất bại so với tổng số yêu cầu \cite{newman2015}. Tỷ lệ lỗi bị ảnh hưởng bởi lỗi mạng, lỗi dịch vụ, timeout hoặc lỗi logic nghiệp vụ \cite{richardson2019}. Mỗi loại lỗi (mạng, timeout, dịch vụ, logic) cần được phân loại và xử lý riêng biệt. Các mẫu giao tiếp khác nhau có cách tiếp cận khác nhau đối với xử lý lỗi, từ HTTP status codes đến dead-letter queues.

Resource Utilization (Sử dụng tài nguyên) đề cập đến lượng tài nguyên hệ thống (CPU, bộ nhớ, băng thông mạng) được sử dụng. Khóa luận giám sát sử dụng tài nguyên cho từng microservice và toàn hệ thống. Các mẫu giao tiếp đồng bộ thường có yêu cầu CPU/bộ nhớ thấp hơn nhưng nhiều kết nối mạng, mẫu bất đồng bộ có thể yêu cầu CPU/bộ nhớ cao hơn nhưng sử dụng mạng hiệu quả hơn.

Scalability (Khả năng mở rộng) đo lường khả năng xử lý tải tăng khi thêm tài nguyên. Khả năng mở rộng theo chiều ngang (thêm instance) thường được ưu tiên hơn chiều dọc (thêm tài nguyên cho instance hiện có). Các mẫu bất đồng bộ thường có khả năng mở rộng tốt hơn do tạo ít phụ thuộc trực tiếp giữa dịch vụ.

Consistency (Tính nhất quán) là khả năng duy trì trạng thái dữ liệu đồng bộ giữa các dịch vụ. Khóa luận đánh giá mức độ nhất quán dữ liệu đạt được bởi các mẫu giao tiếp khác nhau, từ tính nhất quán mạnh (strong consistency) đến nhất quán cuối cùng (eventual consistency).

\subsection{Phương pháp đo lường}
Khóa luận áp dụng nhiều phương pháp bổ sung nhau để thu thập dữ liệu hiệu năng \cite{newman2015}. Load Testing (Kiểm thử tải) mô phỏng điều kiện tải thực tế và đánh giá hiệu năng dưới áp lực \cite{jun2018}. Các kịch bản kiểm thử như kiểm tra tăng dần, chịu tải, phá vỡ và độ bền được thiết kế để mô phỏng trường hợp thực tế (tạo đơn hàng, kiểm tra tồn kho, xử lý thanh toán, gửi thông báo).

Benchmarking (Đánh giá) so sánh hiệu năng của các cấu hình hệ thống khác nhau trong điều kiện tiêu chuẩn \cite{richardson2019}. Benchmark được tiến hành cho mỗi mẫu giao tiếp với các trường hợp thử nghiệm giống nhau, từ 10 đến 100 người dùng đồng thời. Các metric thu thập bao gồm thời gian phản hồi, thông lượng, tỷ lệ lỗi và sử dụng tài nguyên.

Profiling (Lập hồ sơ) phân tích chi tiết tài nguyên và thời gian thực thi của các thành phần. Trong Node.js, profiling thực hiện bằng công cụ như Node.js Profiler hoặc clinic.js. Khóa luận sử dụng profiling để phân tích thời gian cho serialization/deserialization, xử lý mạng, logic nghiệp vụ và tương tác database.

Distributed Tracing (Theo dõi phân tán) theo dõi yêu cầu qua nhiều dịch vụ, xác định điểm nghẽn và mối quan hệ phụ thuộc. OpenTelemetry được tích hợp với NestJS thông qua interceptors và middleware. Mỗi trace đại diện cho một yêu cầu và gồm nhiều spans (hoạt động đơn lẻ như HTTP request, database query).

Metrics Collection (Thu thập số liệu) thu thập và phân tích chỉ số hiệu năng theo thời gian. Khóa luận thu thập HTTP metrics, microservice metrics, database metrics, message broker metrics và system metrics, lưu trữ trong time-series database để phân tích xu hướng và thiết lập cảnh báo.

\subsection{Công cụ đo lường hiệu năng}
Để thực hiện các phương pháp trên, khóa luận triển khai bộ công cụ toàn diện \cite{aksakalli2021}. K6, công cụ kiểm thử tải mã nguồn mở dựa trên JavaScript, tạo tải và đo lường hiệu năng \cite{jun2018}. K6 cho phép viết kịch bản phức tạp mô phỏng hành vi thực tế, hỗ trợ HTTP, WebSocket và gRPC, với khả năng mở rộng và tùy chỉnh cao.

Prometheus, hệ thống giám sát mã nguồn mở, thu thập và lưu trữ số liệu hiệu năng từ microservices \cite{richardson2019}. Prometheus sử dụng mô hình pull để truy vấn định kỳ các mục tiêu được cấu hình \cite{newman2015}, cung cấp ngôn ngữ truy vấn PromQL và hệ thống cảnh báo mạnh mẽ. Các microservices được cấu hình để hiển thị endpoint metrics (/metrics) mà Prometheus truy vấn mỗi 15 giây.

Kết hợp các công nghệ triển khai và công cụ đo lường này tạo môi trường toàn diện để đánh giá và so sánh các mẫu giao tiếp. Thông qua thu thập và phân tích dữ liệu từ nhiều góc độ, khóa luận cung cấp cái nhìn sâu sắc về ưu nhược điểm của mỗi mẫu và đưa ra khuyến nghị dựa trên bằng chứng cho việc lựa chọn mẫu phù hợp trong từng tình huống.
\section{Tổng kết}

Chương này đã cung cấp một cái nhìn tổng quan về kiến trúc microservice và vai trò quan trọng của giao tiếp trong kiến trúc này. Chúng ta đã thảo luận về các đặc điểm chính của microservices, so sánh với kiến trúc nguyên khối, và xem xét các lợi ích cũng như thách thức.

Về mặt đặc điểm, microservice là một kiến trúc phân tán, trong đó mỗi dịch vụ tự trị, tập trung vào một chức năng nghiệp vụ cụ thể, quản lý dữ liệu riêng của nó, được thiết kế để xử lý lỗi, và có thể phát triển độc lập. So với kiến trúc nguyên khối, microservice cung cấp khả năng mở rộng có mục tiêu, phát triển nhanh hơn, tính linh hoạt công nghệ, khả năng chịu lỗi tốt hơn, và khả năng bảo trì và hiểu biết tốt hơn.

Tuy nhiên, microservice cũng đặt ra một số thách thức, bao gồm độ phức tạp phân tán, giao tiếp giữa các dịch vụ, quản lý dữ liệu, vận hành và giám sát, và kiểm thử. Để giải quyết những thách thức này, một số nguyên tắc thiết kế nên được tuân thủ, bao gồm Single Responsibility Principle, Domain-Driven Design, API First, tự động hóa, Monitoring và Observability, và Fault Tolerance.

Trong phần về giao tiếp, chúng ta đã khám phá vai trò quan trọng của giao tiếp trong kiến trúc microservice, bao gồm tạo điều kiện cho sự hợp tác giữa các dịch vụ, đảm bảo tính nhất quán dữ liệu, hỗ trợ khả năng chịu lỗi, và cho phép tính mở rộng. Chúng ta cũng đã thảo luận về các thuộc tính quan trọng của giao tiếp microservice, bao gồm độ tin cậy, độ trễ, khả năng mở rộng, cách ly lỗi, tính nhất quán, định dạng dữ liệu, khả năng tương tác, và bảo mật.

Hai mô hình giao tiếp cơ bản trong microservices là đồng bộ và bất đồng bộ. Trong giao tiếp đồng bộ, người gửi đợi phản hồi từ người nhận, trong khi trong giao tiếp bất đồng bộ, người gửi không đợi phản hồi. Ngoài ra, các microservice cũng giao tiếp theo các kiểu tương tác khác nhau, bao gồm one-to-one, one-to-many, many-to-one, và many-to-many.

Có nhiều công nghệ và giao thức được sử dụng cho giao tiếp microservice, bao gồm HTTP/REST, gRPC, Message Queue, Pub/Sub, WebSockets và GraphQL. Mỗi công nghệ có ưu và nhược điểm riêng và phù hợp với các tình huống khác nhau.

Giao tiếp microservice đặt ra một số thách thức, bao gồm Network Reliability, Service Discovery, Load Balancing, Data Consistency, Versioning, Error Handling, và Monitoring and Debugging. Để giải quyết những thách thức này, một số mẫu giao tiếp đã được phát triển, bao gồm API Gateway, Circuit Breaker, Bulkhead, Retry, Timeout, Saga, Event Sourcing, và CQRS.

Trong phần về đo lường hiệu năng, chúng ta đã thảo luận về các thông số đo lường chính, bao gồm Latency, Throughput, Error Rate, Resource Utilization, và Scalability. Chúng ta cũng đã khám phá các phương pháp đo lường, bao gồm Load Testing, Benchmarking, Profiling, Distributed Tracing, và Metrics Collection. Cuối cùng, chúng ta đã giới thiệu một số công cụ phổ biến để đo lường hiệu năng của microservices, bao gồm K6, Prometheus, Grafana, Jaeger/Zipkin, và ELK Stack.

Các khái niệm và hiểu biết từ chương này sẽ làm nền tảng cho các chương tiếp theo, nơi chúng ta sẽ đi sâu vào việc phân tích chi tiết các mẫu giao tiếp cụ thể trong kiến trúc microservice. Chúng ta sẽ phân loại các mẫu này theo tiêu chí đồng bộ/bất đồng bộ và one-to-one/one-to-many, phân tích ưu và nhược điểm của từng mẫu, và cung cấp hướng dẫn cho việc lựa chọn mẫu phù hợp cho các tình huống cụ thể.

Việc hiểu rõ các khái niệm cơ bản và thách thức của giao tiếp microservice sẽ giúp chúng ta đánh giá tốt hơn hiệu quả của các mẫu giao tiếp trong các kịch bản thực tế. Đồng thời, các phương pháp và công cụ đo lường hiệu năng đã được giới thiệu sẽ được áp dụng trong các phần tiếp theo để đánh giá hiệu suất của các mẫu giao tiếp và đưa ra các khuyến nghị dựa trên dữ liệu.

Tóm lại, chương này đã cung cấp một cái nhìn toàn diện về kiến trúc microservice và vai trò quan trọng của giao tiếp trong kiến trúc này. Chúng ta đã hiểu được các đặc điểm, lợi ích và thách thức của microservice, cũng như các mô hình giao tiếp, công nghệ và mẫu thiết kế phổ biến. Những kiến thức này sẽ là nền tảng vững chắc cho các phân tích chi tiết hơn trong các chương tiếp theo.  