\section{Thảo luận}

\subsection{Lựa chọn mẫu giao tiếp phù hợp cho từng kịch bản}
Dựa trên kết quả đánh giá toàn diện, có thể đưa ra những khuyến nghị về việc lựa chọn mẫu giao tiếp phù hợp trong kiến trúc vi dịch vụ.

Đối với kịch bản kiểm tra và cập nhật tồn kho, giao tiếp bất đồng bộ sử dụng Message Queue (RabbitMQ) được khuyến nghị vì thời gian phản hồi ban đầu nhanh (2.91ms so với 10.38ms), tỷ lệ nhất quán dữ liệu cao (97.2\% so với 93.9\%) và sử dụng tài nguyên hiệu quả (sử dụng CPU ít hơn 84\%). Mặc dù có thời gian xử lý đầu cuối dài hơn một chút, nhưng ưu điểm này được bù đắp bởi khả năng duy trì hiệu suất ổn định dưới tải cao.

Đối với kịch bản xử lý thanh toán, giao tiếp bất đồng bộ sử dụng Message Queue cũng là lựa chọn ưu việt với thời gian phản hồi ban đầu nhanh hơn 99.8\% (2.25ms so với 1508.14ms) và thông lượng cao hơn 52 lần (90.04 msg/s so với 1.68 req/s). Trong bối cảnh thanh toán, việc phản hồi nhanh cho người dùng rằng yêu cầu đã được tiếp nhận là rất quan trọng, ngay cả khi quá trình xử lý thực sự có thể kéo dài.

Đối với kịch bản thông báo kết quả đơn hàng, mô hình Pub/Sub sử dụng Kafka là lựa chọn vượt trội với thời gian broadcast nhanh hơn 97.8\% (11.53ms so với 520.55ms), khả năng phục hồi tốt khi có service gặp lỗi (thời gian phục hồi 2.30ms so với 4793.46ms), và sử dụng tài nguyên hiệu quả (CPU thấp hơn 73.5\%). Mô hình này đặc biệt phù hợp khi một sự kiện cần được xử lý bởi nhiều service độc lập.

Đối với kịch bản ghi nhận hoạt động người dùng, Kafka (mô hình một-nhiều) được khuyến nghị vì hiệu quả sử dụng tài nguyên tốt hơn (CPU thấp hơn 15.8\% và bộ nhớ thấp hơn 10.5\%) và mô hình một-nhiều phù hợp hơn cho việc phân phối dữ liệu hoạt động đến nhiều service phân tích khác nhau.

\begin{table}[H]{Khuyến nghị mẫu giao tiếp phù hợp cho các kịch bản}
    \centering
    {\setlength{\arrayrulewidth}{1pt}
        \renewcommand{\arraystretch}{1.5}
        \setlength{\tabcolsep}{6pt}
        \begin{tabular}{|p{3.2cm}|p{3.2cm}|p{4.6cm}|}
            \hline
            \textbf{Mẫu giao tiếp}         & \textbf{Kịch bản áp dụng}                                       & \textbf{Lý do chính}                                                     \\
            \hline
            Đồng bộ (REST/GraphQL)         & Truy vấn đơn giản, thời gian xử lý ngắn, cần phản hồi trực tiếp & Tính nhất quán dữ liệu tức thì, dễ triển khai, thời gian đầu cuối ngắn \\
            \hline
            Bất đồng bộ một-một (RabbitMQ) & Xử lý thời gian dài, cần phản hồi nhanh cho người dùng          & Phản hồi ban đầu nhanh, quản lý tải hiệu quả, khả năng chịu lỗi tốt      \\
            \hline
            Bất đồng bộ một-nhiều (Kafka)  & Phân phối sự kiện đến nhiều dịch vụ, cần lưu trữ dữ liệu        & Thời gian broadcast nhanh, dễ mở rộng, lưu trữ và phát lại sự kiện       \\
            \hline
        \end{tabular}}
\end{table}

Giao tiếp đồng bộ thích hợp cho những trường hợp cần tính nhất quán dữ liệu tức thì và phản hồi trực tiếp. Mẫu này có thời gian xử lý đầu cuối ngắn hơn và thông lượng cao hơn trong một số tình huống, nhưng tiêu tốn nhiều tài nguyên và khó duy trì hiệu suất khi tải cao. Giao tiếp đồng bộ nên được lựa chọn khi cần đảm bảo tính nhất quán dữ liệu ngay lập tức, thời gian xử lý tác vụ ngắn và đơn giản, client cần phản hồi trực tiếp để tiếp tục quy trình nghiệp vụ, và khối lượng yêu cầu vừa phải và có thể dự đoán được.

Giao tiếp bất đồng bộ dạng một-một mang lại lợi thế về thời gian phản hồi ban đầu nhanh và sử dụng tài nguyên hiệu quả. Mẫu này giúp cải thiện trải nghiệm người dùng, đồng thời duy trì hiệu suất ổn định dưới tải cao. Mặc dù có thời gian xử lý đầu cuối dài hơn một chút, nhưng ưu điểm này được bù đắp bởi khả năng chịu lỗi tốt hơn. Giao tiếp bất đồng bộ dạng một-một nên được lựa chọn khi cần phản hồi nhanh cho người dùng, thời gian xử lý tác vụ kéo dài, cần độ tin cậy cao trong xử lý message, cần quản lý tải và phân phối công việc hiệu quả, và hệ thống cần khả năng chịu lỗi và độ ổn định cao.

Giao tiếp bất đồng bộ dạng một-nhiều thể hiện hiệu suất vượt trội trong các tình huống cần phân phối sự kiện đến nhiều service. Mẫu này có thời gian broadcast nhanh, sử dụng tài nguyên hiệu quả, và khả năng phục hồi tuyệt vời khi có service gặp lỗi. Đặc biệt, mẫu này rất dễ mở rộng khi thêm các service xử lý mới. Giao tiếp bất đồng bộ dạng một-nhiều nên được lựa chọn khi một sự kiện cần được xử lý bởi nhiều service độc lập, cần khả năng lưu trữ và phát lại các sự kiện, hệ thống yêu cầu khả năng mở rộng cao, cần xử lý dữ liệu theo thời gian thực với thông lượng lớn, và các service xử lý có thể bị thêm hoặc xóa linh hoạt.

\subsection{Mô hình tích hợp các mẫu giao tiếp}
Trong thực tế, việc kết hợp các mẫu giao tiếp khác nhau thường mang lại kết quả tốt nhất cho hệ thống vi dịch vụ. Mỗi mẫu giao tiếp có những ưu điểm và nhược điểm riêng, phù hợp với những tình huống cụ thể. Một mô hình tích hợp hiệu quả có thể tận dụng ưu điểm của từng phương pháp để tối ưu hóa hiệu suất tổng thể của hệ thống.

Một chiến lược tích hợp hiệu quả là sử dụng giao tiếp đồng bộ cho các tác vụ yêu cầu phản hồi tức thời và có thời gian xử lý ngắn, như truy vấn thông tin đơn giản. Giao tiếp bất đồng bộ dạng một-một được sử dụng cho các tác vụ có thời gian xử lý dài nhưng cần phản hồi nhanh cho người dùng, như xử lý đơn hàng và thanh toán. Giao tiếp bất đồng bộ dạng một-nhiều được sử dụng cho các tác vụ cần phân phối thông tin đến nhiều service, như thông báo sự kiện và ghi nhận hoạt động.

Trong một hệ thống thương mại điện tử, mô hình tích hợp này có thể được triển khai như sau: REST API được sử dụng cho việc hiển thị thông tin sản phẩm và danh mục, RabbitMQ được sử dụng cho xử lý đơn hàng và thanh toán, và Kafka được sử dụng cho thông báo kết quả đơn hàng và phân tích dữ liệu.

Một yếu tố quan trọng cần xem xét khi thiết kế mô hình tích hợp là độ phức tạp của hệ thống. Việc sử dụng nhiều mẫu giao tiếp khác nhau có thể làm tăng độ phức tạp trong triển khai và bảo trì. Do đó, cần cân nhắc giữa lợi ích hiệu suất và chi phí phức tạp khi quyết định số lượng mẫu giao tiếp cần sử dụng.

\subsection{Tối ưu hóa hiệu suất trong thực tế}
Ngoài việc lựa chọn mẫu giao tiếp phù hợp, còn có nhiều chiến lược tối ưu hóa hiệu suất khác có thể áp dụng trong thực tế. Dựa trên kết quả đánh giá, có thể đưa ra một số khuyến nghị sau:
Đối với giao tiếp đồng bộ, việc triển khai mẫu Circuit Breaker là rất quan trọng để ngăn chặn lỗi lan truyền và cải thiện khả năng chịu lỗi của hệ thống. Mẫu này giúp ngăn chặn các yêu cầu đến dịch vụ không khả dụng, giảm thiểu tác động của lỗi dịch vụ đến toàn bộ hệ thống. Việc sử dụng thời gian chờ hợp lý cũng giúp tránh tình trạng chờ đợi vô hạn khi dịch vụ gặp sự cố.

Đối với giao tiếp bất đồng bộ, việc điều chỉnh kích thước hàng đợi và số lượng tiến trình xử lý có thể giúp cân bằng giữa thông lượng và độ trễ. Tăng số lượng tiến trình xử lý giúp cải thiện thông lượng, nhưng cũng làm tăng chi phí tài nguyên. Việc triển khai cơ chế thử lại với thời gian chờ tăng dần giúp xử lý các lỗi tạm thời, trong khi hàng đợi thông điệp lỗi giúp xử lý các thông điệp không thể xử lý.

Đối với mô hình Pub/Sub, việc phân vùng dữ liệu có thể giúp cải thiện khả năng mở rộng và hiệu suất. Việc chọn số lượng partition phù hợp với số lượng consumer giúp tối ưu hóa cân bằng tải và thông lượng. Việc duy trì kích thước thông điệp nhỏ và sử dụng định dạng nhị phân như Avro hoặc Protobuf thay vì JSON cũng giúp cải thiện hiệu suất.