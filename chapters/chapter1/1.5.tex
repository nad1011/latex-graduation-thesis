\section{Ý nghĩa khoa học và thực tiễn}
Khóa luận này mang lại đóng góp giá trị cả về khoa học và ứng dụng thực tiễn trong lĩnh vực kiến trúc phần mềm và hệ thống phân tán.

Về mặt khoa học, khóa luận hệ thống hóa các mẫu giao tiếp trong kiến trúc microservice và xây dựng khung phân tích toàn diện cho việc đánh giá chúng. Qua việc tổng hợp tài liệu học thuật, khóa luận xác định được các nguyên tắc cơ bản và yếu tố ảnh hưởng đến hiệu quả của từng mẫu giao tiếp, tạo nền tảng lý thuyết cho nghiên cứu tương lai. Khóa luận cũng phát triển phương pháp đánh giá mới kết hợp các tiêu chí định lượng và định tính, giúp các nhà nghiên cứu đưa ra quyết định dựa trên dữ liệu thực nghiệm.

Về mặt thực tiễn, khóa luận cung cấp hướng dẫn cụ thể cho việc lựa chọn và triển khai các mẫu giao tiếp trong dự án thực tế. Thông qua phân tích trường hợp thực tế, khóa luận rút ra các nguyên tắc và thực tiễn tốt nhất, giúp nhà phát triển áp dụng các mẫu giao tiếp hiệu quả và tránh lỗi phổ biến. Ứng dụng microservice mẫu và các kịch bản thử nghiệm được phát triển trong khóa luận cho phép các nhà phát triển đánh giá các mẫu giao tiếp trong môi trường kiểm soát trước khi triển khai thực tế, giảm thiểu rủi ro và tăng cường sự tự tin.

Ngoài ra, khóa luận nâng cao nhận thức về các mẫu giao tiếp trong cộng đồng phát triển, khuyến khích chia sẻ kiến thức và thúc đẩy phát triển công cụ hỗ trợ triển khai mẫu giao tiếp hiệu quả. Điều này đóng góp vào việc xây dựng hệ sinh thái phong phú cho phát triển và triển khai hệ thống microservice.