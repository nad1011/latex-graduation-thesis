Trong chương này, nội dung đã trình bày chi tiết về các mẫu giao tiếp trong kiến trúc vi dịch vụ. Đầu tiên, các mẫu giao tiếp được phân loại theo hai tiêu chí chính: phương thức giao tiếp và phạm vi giao tiếp.

Đối với giao tiếp đồng bộ một đối một, trọng tâm được đặt vào mẫu Request/Response, được triển khai thông qua các công nghệ phổ biến như REST, gRPC và GraphQL. Mẫu này có ưu điểm là đơn giản, trực quan và phản hồi tức thì, tuy nhiên tồn tại hạn chế về khả năng mở rộng và độ tin cậy.

Trong giao tiếp bất đồng bộ một đối một, các mẫu như One-way Notifications và Message Queue đã được xem xét. Những mẫu này cung cấp mức độ tách rời tốt hơn, khả năng chịu lỗi cao và khả năng mở rộng tốt, nhưng lại có độ phức tạp cao hơn trong quá trình triển khai và gỡ lỗi.

Đối với giao tiếp bất đồng bộ một đối nhiều, các mẫu như Publish/Subscribe, Event Sourcing và Message Broker với Exchange Routing đã được phân tích. Những mẫu này cung cấp mức độ tách rời cao nhất và khả năng mở rộng tối ưu, đặc biệt phù hợp cho việc phân phối thông tin và xử lý sự kiện.

Mỗi mẫu giao tiếp đều phù hợp với những tình huống sử dụng cụ thể. Việc lựa chọn mẫu giao tiếp phù hợp phụ thuộc vào nhiều yếu tố như yêu cầu về độ trễ, tính nhất quán dữ liệu, khả năng mở rộng và độ phức tạp khi triển khai.

Trên thực tế, một hệ thống vi dịch vụ hiệu quả thường kết hợp nhiều mẫu giao tiếp khác nhau, áp dụng từng mẫu phù hợp với từng ngữ cảnh cụ thể. Việc hiểu rõ các mẫu giao tiếp cùng với các tiêu chí lựa chọn là nền tảng quan trọng để thiết kế một kiến trúc vi dịch vụ thành công.

Ở chương tiếp theo, quá trình triển khai thử nghiệm các mẫu giao tiếp đã phân tích sẽ được tiến hành trong một hệ thống vi dịch vụ thực tế. Một ứng dụng mẫu với nhiều vi dịch vụ sẽ được xây dựng, áp dụng các mẫu giao tiếp khác nhau và thực hiện đánh giá về hiệu suất cũng như tính phù hợp của từng mẫu trong các tình huống cụ thể.