\section{Tổng kết}

Chương này đã cung cấp tổng quan về kiến trúc vi dịch vụ và vai trò quan trọng của giao tiếp trong kiến trúc này. Vi dịch vụ được đặc trưng bởi tính tự trị cao, tập trung vào chức năng nghiệp vụ, quản lý dữ liệu phi tập trung, thiết kế hướng lỗi và khả năng tiến hóa độc lập. So với kiến trúc nguyên khối, vi dịch vụ mang lại khả năng mở rộng có mục tiêu, phát triển nhanh hơn, linh hoạt công nghệ và khả năng chịu lỗi tốt hơn.

Tuy nhiên, kiến trúc này cũng đặt ra các thách thức về độ phức tạp phân tán, giao tiếp giữa các dịch vụ, quản lý dữ liệu và vận hành. Để giải quyết những thách thức này, các nguyên tắc thiết kế như Single Responsibility, Domain-Driven Design, API First, tự động hóa và Fault Tolerance cần được áp dụng.

Trong vi dịch vụ, giao tiếp đóng vai trò then chốt thông qua hai mô hình cơ bản: đồng bộ (người gửi đợi phản hồi) và bất đồng bộ (người gửi không đợi phản hồi), cùng với các kiểu tương tác một đối một và một đối nhiều. Các mẫu giao tiếp phổ biến bao gồm Request-Response, Event-Driven, Publish-Subscribe và Point-to-Point Messaging, mỗi mẫu phù hợp cho các tình huống khác nhau.

Trong phần về công nghệ triển khai, Khóa luận đã khám phá các công nghệ hiện đại được sử dụng để xây dựng hệ thống vi dịch vụ, NestJS, TypeScript, TypeORM và PostgreSQL cung cấp nền tảng vững chắc cho phát triển vi dịch vụ, trong khi HTTP/REST, RabbitMQ và Kafka hỗ trợ các mẫu giao tiếp khác nhau. Các phương pháp đo lường hiệu năng như Load Testing, Benchmarking, Profiling và Distributed Tracing, cùng với công cụ K6 và Prometheus, giúp đánh giá toàn diện hiệu quả của các mẫu giao tiếp trong môi trường thực tế.